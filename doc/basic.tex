\documentclass[a4paper,12pt]{article}
\usepackage{graphicx}
\date{}
\begin{document}

\section{Overview}
\label{ov}
\begin{description}
\item [StGermain] Handles low level operations, e.g. memory managment.
\item [StgFEM] Provides interfaces and implementations to finite element based
  operations.
\item [PICellerator] Provides routines for handling particle-based integration.
  This includes such things as population control and parallel management.
\item [Underworld] Primarily constituted by material rheologies for use with
  the underlying components.
\item [gLucifer] A visualisation utility providing parallel rendering, allowing
  for visualisation of large data sets.
\end{description}
See \ref{com} for a more complete list of functionality.

\subsection{StGermain}
\label{ov:stgermain}
StGermain provides a suite of components and utilities used as a foundation for
the other higher level packages (e.g. StgFEM). The functionality provided by
StGermain is suited primarily to high-performance scientific applications. This
includes (but not limited to):
\begin{itemize}
\item memory management/profiling, 
\item basic container classes (e.g. hash tables, binary trees), 
\item XML input manipulation, 
\item mesh representation and construction, 
\item basic particle managment routines and
\item low level parallelisation in interfaces.
\end{itemize}

\subsection{StgFEM}
\label{ov:stgfem}
StgFEM provides a set of routines more focused on finite element applications.
\begin{itemize}
\item element types (e.g. linear, bilinear, etc.)
\item parallel equation number calculation, 
\item linear algebra interfaces, 
\item matrix and vector interfaces
\item multigrid solvers, 
\item a set of existing solvers (e.g. advection/diffusion, energy)
\end{itemize}

\subsection{PICellerator}
\label{ov:pic}
\begin{itemize}
\item storage of material properties on particles, 
\item population control (e.g. particle splitting and removal), 
\item particle based integration routines
\end{itemize}

\subsection{Underworld}
\label{ov:underworld}
Going to chat to the guys at Monash for an accurate desciription of this one.

\subsection{gLucifer}
\label{ov:glu}
Same as above; I don't really know what each part of gLucifer is for.

\section{Coding Conventions}
\label{cc}
Describe the phases we use (construct, build, initialise, execute).

Describe entry points, pulgins, components, objects.

Detail the code layout, e.g. src and test directories

The major components discussed thus far all 
Describe the object oriented C style of coding. Include conventions like
underbar prefix for private functions, etc.

\end{document}
